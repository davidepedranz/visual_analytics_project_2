\documentclass[paper=a4, fontsize=11pt]{scrartcl}

\usepackage{url}    
\usepackage[margin=1in]{geometry}
\setlength{\parskip}{0.5em}

% images
\usepackage{graphicx}
\usepackage{float}

% sub-figures
\usepackage{subcaption}

% cref command
\usepackage[noabbrev,capitalise]{cleveref}

% tables
\usepackage{array, booktabs}
\usepackage{longtable}

\usepackage[T1]{fontenc}
\usepackage{fourier}
\usepackage[english]{babel}
\usepackage{amsmath,amsfonts,amsthm}

\usepackage{sectsty} % Allows customizing section commands
\allsectionsfont{\centering \normalfont\scshape} % Make all sections centered, the default font and small caps

\usepackage{fancyhdr} % Custom headers and footers
\pagestyle{fancyplain} % Makes all pages in the document conform to the custom headers and footers
\fancyhead{} % No page header - if you want one, create it in the same way as the footers below
\fancyfoot[L]{} % Empty left footer
\fancyfoot[C]{} % Empty center footer
\fancyfoot[R]{\thepage} % Page numbering for right footer
\renewcommand{\headrulewidth}{0pt} % Remove header underlines
\renewcommand{\footrulewidth}{0pt} % Remove footer underlines
% \setlength{\headheight}{1pt} % Customize the height of the header

\numberwithin{equation}{section} % Number equations within sections (i.e. 1.1, 1.2, 2.1, 2.2 instead of 1, 2, 3, 4)
\numberwithin{figure}{section} % Number figures within sections (i.e. 1.1, 1.2, 2.1, 2.2 instead of 1, 2, 3, 4)
\numberwithin{table}{section} % Number tables within sections (i.e. 1.1, 1.2, 2.1, 2.2 instead of 1, 2, 3, 4)

\setlength\parindent{0pt} % Removes all indentation from paragraphs - comment this line for an assignment with lots of text


%----------------------------------------------------------------------------------------
%  TITLE
%----------------------------------------------------------------------------------------

\newcommand{\horrule}[1]{\rule{\linewidth}{#1}} % Create horizontal rule command with 1 argument of height

\title{	
    \normalfont
    \large \textsc{University of Groningen \\ Visual Analytics for Big Data} \\ [22pt]
    \horrule{0.5pt} \\[0.4cm]
    \huge Practical Assignment: \\ Visualizing Data with Tableau \\
    \horrule{0.5pt} \\[0.5cm]
}

\author{
    Davide Pedranz \\
    \small d.pedranz@student.rug.nl \\
    \small Student Number: S3543757    
}

\date{\vspace{0.5cm} \normalsize\today}

%----------------------------------------------------------------------------------------
%  DOCUMENT
%----------------------------------------------------------------------------------------

\begin{document}
\maketitle

\section*{Abstract}
The scope of this assignment is to answer a number of non trivial question using visualizations built with Tableau.
The dataset contains incidents reported to the police of Seattle in the period $2009$ to $2017$ and is available online at \url{http://data.seattle.org} under the name of \textbf{Seattle Police Department 911 Incident Response}.
We use a dump of the dataset of September 17.

\tableofcontents

\section*{Tableau Visualization}
All visualization used to answer the questions in this report can be found in Tableau Public at the following address:
\url{https://public.tableau.com/profile/davide.pedranz#!/vizhome/seattle_police_department_911_incident_response/InitialvsFinalType}.

\clearpage

% \section{Exploring the incidents' geographical distribution}

\subsection*{Question 1.1}
\textit{Are there high variations of incidents densities over Seattle? Which are the low-density zones? Which are the high density zones?}

To answer this questions, 

\subsection*{Question 1.2}
\textit{Are there high variations of incidents densities over Seattle for specific types of incidents? Which are these types and which are the variations you found?}

% \section{Exploring the incidents' geographical and temporal distribution}

\subsection*{Question 2.1}
\textit{Do you see a different spatial distribution of incidents over the different years? If so, which are the differences you found?}

Before trying to answer this question, it is interesting to take a closer look to the data.
The data span a temporal period of $9$ years, from $2009$ to $2017$.
The dataset contains $1.441.208$ records, of which $1.027.546$ (about $70\%$) have no information about the time of the incident.
Also, there are only $7$ incidents for year $2009$ and $84$ for year $2010$ (we suspect that many incidents without the time information have happened in these years, but we have no evidence of that).
Our analysis concentrate thus on years $2011$ to $2017$.

Similar to Question 1.2, we use the small multiple design technique.
The visualization is composed of multiple maps: each map shows the incidents in Seattle for a single year.
The maps are sorted in chronological order, i.e. $2011$ is the first on the left, $2017$ the last on the right.
The user is free to change the order to make accurate comparison between pairs of years, if needed.
Since the visualization only shows data for $7$ years, it fits in a single screen.

We use a small size for each point on the map and a medium level of transparency to be able to distinguish areas with different densities.
We use color to overload the encoding of the year:
this allows to combine the visualization with the histogram of incidents' frequency by year, as shown in \cref{fig:2_1_geographical_temporal_distribution} (see dashboard \textit{Incidents geographical and temporal distribution} in Tableau).

\begin{figure}[h]
	\centering
	\includegraphics[width=\columnwidth]{figures/2_1_geographical_temporal_distribution}
	\caption{Incidents spatial distribution over different years in Seattle.}
	\label{fig:2_1_geographical_temporal_distribution}
\end{figure}

From the visualization, we can not notice any significant difference in the spatial distribution of incidents over the last $5$ years.
Year $2013$ has significantly less incidents, but this is probably due to the missing temporal information on $70\%$ of the entries.
Anyway, their distribution is similar to the general one: incidents are more concentrated in the city center (see Question 1.1).


\subsection*{Question 2.2}
\textit{Are there zones with a consistent low incidents density over all years? Are there zones with a consistent high density over all years?}

We can use the visualization built for the previous question to answer this one.
Over all years, the city center has a higher concentration of incidents.
Outside the city center, incidents seems to be slightly more concentrated along the main streets.
The distribution in each year is similar to the distribution over all years and is described in greater details in \cref{sec:question1}.

% \section{Exploring the resolution speed}

\subsection*{Question 3.1}
\textit{What is the average resolution speed to an incident? Resolution speed is defined as the difference between the Event Clearance Date value (moment when the police closed the file of an incident) and the At Scene Time value (moment when the police arrived at the scene of the incident).}

The dataset does not define the average resolution time explicitly.
Thus, we need to add a new columns to the dataset with this information.
Tableau is able to compute new dimensions using the existing ones.
Since we compute a difference between $2$ timestamp, we use the function ``datediff'' to get a proper result.

The average resolution time of the entire dataset is about $2$ hours.

\begin{figure}[h]
	\centering
	\includegraphics[width=0.9\columnwidth]{figures/3_1_resolution_speed_outliers}
	\caption{Distribution of the resolution times. The annotations on the plot show some outliers with very high resolution times. Most of the incidents are closed in few hours.}
	\label{fig:3_1_resolution_speed_outliers}
\end{figure}

\cref{fig:3_1_resolution_speed_outliers} shows the distribution of resolution times.
We can notice that:
\begin{itemize}
    \item Most incidents are closed in few hours.
    \item There are few outliers that have a resolution time of over $1500$ hours (approximately 2 months). All outliers share the same type: ``Car Prowl''.
\end{itemize}

The line chart gives already some interesting insides about incidents resolution times.
However, due to the described outliers, it is difficult to read the left part of the chart, where most of the distribution mass is.

To solve the problem, we add an histogram of the resolution times.
Resolution time is a continuos measure.
However, we are not interested in the precise value of resolution time for each incident, rather in the overall distribution.
To visualize the distribution, we discretize the resolution time in buckets of $1$ hour.
Since most of the density mass in the first $15$ buckets, we cut the histogram after the first $20$ ones.
Since we are loosing the outliers, we add a table that contains such information.

\cref{fig:3_1_resolution_speed} shows the final dashboard.
We can notice that:
\begin{itemize}
    \item The average resolution time is about $2$ hours.
    \item There are some outliers, but their number is really low in comparison to the amount of entries.
    \item Overall most incidents are closed in less than $10$ hours.
\end{itemize}

\begin{figure}[h]
	\centering
	\includegraphics[width=\columnwidth]{figures/3_1_resolution_speed}
	\caption{Distribution of the resolution times. The dashboard is called ``Resolution Times'' in Tableau.}
	\label{fig:3_1_resolution_speed}
\end{figure}


\subsection*{Question 3.2}
\textit{Are there certain types of incidents having a much lower resolution speed than others? If so, which are these?}

The visualization uses a bar chart.
Each type of is encoded in one bar and the height of the bar corresponds to the average resolution time for that type.
The types are ordered by average resolution time decreasing, so it is easy to spot the outliers.
Color is used to overload the encoding of the incident type.
The bar chart show the average line to make comparison and reasoning easier.
We prefer to use the x-axis for the type and the y-axis for the average resolution time since the visualization with swapped axises does not fit a single screen.

\cref{fig:3_2_resolution_speed_by_type} shows the visualization.
We can notice that:
\begin{itemize}
    \item ``Homicide'' is the type with the highest average resolution time. This is not a surprise, since such an incident requires very accurate investigations.
    \item ``Public Gatherings'' has the seconds highest average resolution time. This is probably due to the high number of people involved in the incident.
    \item ``Vice Calls'' is has the fastest resolution time, on average only about $30$ minutes.
    \item ``False Alarms'' are also resolved quite fast, less than $1$ hour on average.
    \item All other types have resolution times between $1$ and $3$ hours on average.
    \item The distribution is skewed to towards types with higher resolution times.
\end{itemize}

\begin{figure}[h]
	\centering
	\includegraphics[width=\columnwidth]{figures/3_2_resolution_speed_by_type}
	\caption{Incidents' resolution time by type. The sheet is called ``Resolution Time by Type'' in Tableau.}
	\label{fig:3_2_resolution_speed_by_type}
\end{figure}


\subsection*{Question 3.3}
\textit{Does the resolution speed depend on the time period (e.g., year, season of the year)?}

% \section{Exploring the incidents' classification}

\section{Exploring the incidents' temporal distribution}

\subsection*{Question 5.1}
\textit{How do the different types of incidents (as represented by Event Clearance Group) vary in number over the different years? Do you see different types of incidents having the same temporal variation pattern over the same year(s)?}


\subsection*{Question 5.2}
\textit{How do the different types of reported incidents (as represented by Initial Type Group) vary over the 24 hours of a day, for the entire data collection? Are there certain hours having a higher rate of reported incidents?}


\subsection*{Question 5.3}
\textit{How long does it take to clear incidents, as a function of the hour when they were reported? For example, do incidents reported at noon get cleared faster than incidents reported in the middle of the night?}

The visualization uses a bar chart.
The x-axis represents the hour of the day, while the y-axis shows the average resolution time for incidents reported in the given hour.
Color overloads the value showed in the y-axis.
The average line helps to compare particular time slots with the average behaviour.

Best practices for visualization suggests to use line charts instead of bar charts for continuous values like time.
In this case, however, we have discretized time into buckets of one hour and aggregated them together to compute the average.
In other words, we treat time as a discrete ordinal value.

\begin{figure}[h]
	\centering
	\includegraphics[width=0.9\columnwidth]{figures/5_3_resolution_speed_vs_hour}
	\caption{Resolution speed of incidents reported in different hour. The sheet is called \textit{Resolution Speed vs Day Hour} in Tableau.}
	\label{fig:5_3_resolution_speed_vs_hour}
\end{figure}

\cref{fig:5_3_resolution_speed_vs_hour} shows the visualization.
We can notice that:
\begin{itemize}
    \item The average time required to close incidents is not constant over the different hours of the day.
    \item Incidents that happens in the slots $11 - 14$ and $19 - 20$ take significant longer to be closed.
    \item Incidents that happens in the early morning ($5 - 7$) require on average more time than the incidents in the night and late morning ($9 - 10$).
    \item The time slots between $5 - 7$, $11 - 14$ and $19 - 20$ have a faster resolution speed.
\end{itemize}

To sum up, there is a correlation between the hour of the day and the time required to close incidents.
The visualization fully answers the question.

We have created a second visualization to check if this behaviour is common to all years.
The visualization uses again bar charts, one for each year.
The last column shows the average over all years as a reference for comparisons.
The axises are swapped in comparison to \cref{fig:5_3_resolution_speed_vs_hour} to better fit the screen.

\begin{figure}[h]
	\centering
	\includegraphics[width=0.9\columnwidth]{figures/5_3_resolution_speed_vs_hour_by_year}
	\caption{Resolution speed of incidents reported in different hour by year. The sheet is called \textit{Resolution Speed vs Day Hour (by year)} in Tableau.}
	\label{fig:5_3_resolution_speed_vs_hour_by_year}
\end{figure}

From \cref{fig:5_3_resolution_speed_vs_hour_by_year} we can notice that:
\begin{itemize}
    \item In $2011$ incidents are closed much faster than the average. There are some hours with a slightly higher resolution speed ($9$, $14$, $19$), but the difference with the rest of the day is small.
    \item Incidents that happens at $20$ in year $2012$ seems to take much longer to close than the average. This is probably due to the outliers discussed in \cref{sec:question3}, since both have very long resolution times and are in that slot of time.
    \item Years $2013$ have some picks in the slots $12 - 13$ and $18 - 20$. However, there are probably missing data for this year (see \cref{sec:question1}, so the confidence for this observation is low.
    \item Years $2014$, $2015$, $2016$ and $2017$ have a behaviour identical to the average over all years.    
\end{itemize}

To sum up, the resolution speed for the different hours of the day seems to be stable over years $2014$ to $2017$.
The previous years have slightly different behaviour, but still share some patterns.
Incidents that occurs in slots $5 - 7$, $11 - 14$ and $19 - 20$ take more time to be closed, while incidents between this slots are closed much faster.


% \bibliographystyle{IEEEtran}
% \bibliography{references}

\end{document}

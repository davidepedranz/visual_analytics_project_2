\section{Exploring the incidents' geographical distribution}

\subsection*{Question 1.1}
\textit{Are there high variations of incidents densities over Seattle? Which are the low-density zones? Which are the high density zones?}

The visualization uses a map of the city of Seattle (a scatterplot that uses the maps from OpenStreetMap for the background):
each point in the dataset is visualized as a circle on the map.
Columns and rows are used to encode respectively longitude and latitude.
The size of the circles is set to the minimum, the opacity to the value $8\%$.
Size and opacity does not encode any particular information, but the combination of small size, opacity and zoom level allows to visualize an approximation of the density of incidents.
The zoom level is small in order to get an overview of the entire city that fits the screen, size and opacity are regulated accordingly.

\begin{figure}[h]
	\centering
	\includegraphics[width=.75\columnwidth]{figures/1_1_geographical_distribution_incidents}
	\caption{Geographical distribution of the incidents in Seattle. The screenshot is taken from the sheet called \textit{Geographical Distribution of Incidents in Seattle} in Tableau.}
	\label{fig:1_1_geographical_distribution_incidents}
\end{figure}

\cref{fig:1_1_geographical_distribution_incidents} shows that there are indeed variations in the density of incidents in Seattle:
\begin{itemize}
    \item The regions in the very north and south of the city have a very small number of incidents. These areas are already outside the borders of Seattle. We suspect they are not usually covered by the police of Seattle, so the dataset may not contain all the incidents that occurred in the those areas.
    \item Incidents seems to be more concentrated in the central area of the city, around the Elliott Bay, and immediately to its north.
    \item Outside the city center, the main roads have a higher number of incidents than the surrounding areas (in particular in the south part of the city).
    \item The density of incidents in the rest of the city is lower and seems to be approximately uniform.
\end{itemize}

The visualization answers the question, but it is not optimal.
A better solution would be to use a density plot which is able to visualize better the differences of incidents' densities in the city.
Unfortunately, Tableau does offer density plot: the proposed solution tries to approximate a density plot in Tableau.

\subsection*{Question 1.2}
\textit{Are there high variations of incidents densities over Seattle for specific types of incidents? Which are these types and which are the variations you found?}

The visualization uses an approach similar to the previous section, but uses pagination to display the different types of incidents in isolation.
This is done by using the ``Pages'' option in Tableau:
it is possible to interactively change the type visualized by using the dedicated menu.
The visualization uses a filter to exclude the observations without a type.
Since there are less observations in each page, the opacity value is set to a higher value ($30\%$).

\begin{figure}[h] 
    \begin{subfigure}{0.5\textwidth}
        \includegraphics[width=0.9\linewidth]{figures/1_2_geographical_distribution_arrests} 
        \caption{Arrests}
        \label{fig:1_2_arrests}
    \end{subfigure}
    \begin{subfigure}{0.5\textwidth}
        \includegraphics[width=0.9\linewidth]{figures/1_2_geographical_distribution_prostitution}
        \caption{Prostitution}
        \label{fig:1_2_prostitution}
    \end{subfigure}
    \caption{Geographical distribution of the incidents of type \textit{arrest} (on the left) and \textit{prostitution} (on the right) in Seattle. The screenshots are taken from the sheet called \textit{Sheet 2} in Tableau.}
    \label{fig:1_2_geographical_distribution_by_type}
\end{figure}

\cref{fig:1_2_geographical_distribution_by_type} shows $2$ snapshots for the types ``arrest'' and ``prostitution''.
From the figure it is already possible to notice that different types of incidents have different distributions in the city.
\cref{tab:distribution_by_type} contains the observations for every type of incident.

\renewcommand{\arraystretch}{1.5}
\begin{longtable}{ | >{\arraybackslash} m{3.8cm} | >{\arraybackslash} m{11.2cm} | }
    \hline
    \textbf{Incident Type} & \textbf{Description} \\
    \hline
    Accident Investigation  &   Higher concentration in the city center. Outside the city center, main roads have a higher concentration than the surrounding areas. \\
    \hline        
    Animal Complaints       &   Slightly higher concentration in the city center. \\
    \hline
    Arrest                  &   Higher concentration in the city center and along main roads in the north and south of the city. \\
    \hline
    Assaults                &   Like ``Arrest''. \\
    \hline
    Auto Thefts             &   No significant variations. \\
    \hline
    Behavioral Health       &   Higher concentration in the city center. \\
    \hline
    Bike                    &   Higher concentration in the city center. The north has a higher concentration thant the south. \\
    \hline
    Burglary                &   No significant variations. \\
    \hline
    Car Prowl               &   Slightly more concentrated in the city center. \\
    \hline
    Disturbances            &   Like ``Car Prowl''. \\
    \hline
    Drive By (No Injury)    &   Higher concentration in south east, almost absent in the other sectors. \\
    \hline
    Failure to Register (Sex Offender) &   Slightly more concentrated in the city center and south east. \\
    \hline
    False Alacad            &   No significant variations. \\
    \hline
    False Alarms            &   No significant variations. \\
    \hline
    Fraud Calls             &   Like ``Car Prowl''. \\
    \hline
    Harbor Calls            &   Concentrated along the coast. \\
    \hline
    Hazards                 &   Like ``Arrest''. \\
    \hline
    Homicide                &   Slightly more concentrated in the city center. \\
    \hline
    Lewd Conduct            &   Higher concentration in the city center. \\
    \hline
    Liquor Violations       &   Like ``Arrest''. \\
    \hline
    Miscellaneous Misdemeanors &   Like ``Arrest''. \\
    \hline
    Motor Vehicle Collision Investigation &   More concentrated in the center and along the streets. \\
    \hline
    Narcotics Complaints    &   Like ``Arrest''. \\
    \hline
    Nuisance, Mischief      &   Like ``Arrest''. \\
    \hline
    Other Property          &   Like ``Arrest''. \\
    \hline
    Person Down / Injury    &   Higher concentration in the city center. \\
    \hline
    Persons - Lost, Found, Missing &   Slightly more concentrated in the city center. \\
    \hline
    Property - Missing, Found    &   Slightly more concentrated in the city center. \\
    \hline
    Property Damage         &   Like ``Arrest''. \\
    \hline
    Prostitution            &   Concentrated along the main streets in the north and in the south. Present also in the city center. \\
    \hline
    Prowler                 &   No significant variations. \\
    \hline
    Public Gatherings       &   Higher concentration in the city center. \\
    \hline
    Reckless Burning        &   No significant variations. \\
    \hline
    Robbery                 &   Like ``Arrest''. \\
    \hline
    Shoplifting             &   Like ``Arrest''. \\
    \hline
    Suspicious Circumstances &   No significant variations. \\
    \hline
    Threats, Harassment     &   Like ``Arrest''. \\
    \hline
    Traffic Related Calls   &   No significant variations. \\
    \hline
    Trespass                &   Like ``Arrest''. \\
    \hline
    Vice Calls              &   No significant variations. \\
    \hline
    Weapon Calls            &   Like ``Arrest''. \\
    \hline

    \caption{Distribution of incidents in Seattle by type.}
    \label{tab:distribution_by_type}
\end{longtable}
